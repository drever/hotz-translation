\documentclass{article}
    % General document formatting
    \usepackage[margin=0.7in]{geometry}
    \usepackage[parfill]{parskip}
    \usepackage[utf8]{inputenc}

    \usepackage{graphicx}    
    % Related to math
    \usepackage{amsmath,amssymb,amsfonts,amsthm}

\begin{document}

\title{Electronic information processing and cybernetics - An algebrasation of the synthesis problem for circuits}
\author{Günter Hotz}
\maketitle

\section{Introduction}
The occasion for this work is a problem from automata theory: From a given set of building blocks an automaton, whose functionality is predetermined, shall be assembled. From the different, eventually existing  solutions the cheapest shall be selected. 

A building block $A \in \mathcal{U}$ is a physical, mostly electrical device with $Q(A)$ inputs and $Z(A)$ outputs. For each input a Set $S$ of input signals is permitted, on which the building block reacts with output signals. We assume the following simplifications with regard to the issue, the following holds:

\begin{enumerate}
\item For each input of the elemnts of $\mathcal{U}$ a set of signals $S$ is prescribed, and each element of $S^n$ is allowed as input signal for $A$ with $n = Q(A)$.
\item The set of output signals of $A \in \mathcal{U}$ lies in $S^m$ with $m = Z(A)$.
\item If at time $t$ the input signal $s \in S^n$ is applied to $A$, then the output signal at time $t$ is uniquely determined by $s$. (We therefore neglect the finite propagation speed of signals).
\end{enumerate}

Thus, the finite automaton is completely described by its function $\phi(A):S^n \rightarrow S^m$. It is presumed that inputs and outputs of $A$ are labeled with a fixed numbering from $1$ to $Q(A)$, and repesctively from $1$ to $Z(A)$. The $i$-th input (output) is assigned to the $i$-th component of $S^n$ ($S^m$). 

An element of $\mathcal{U}$ is a circuit. If $A$ and $B$ are circuits with $Q(A)$, or $Q(B)$ inputs and $Z(A)$, or repectivley $Z(B)$ outputs, then we build new circuits from $A$ and $B$ by integrating them to a new element $A\times B$ with $Q(A)+Q(B)$ inputs and $Z(A)+Z(B)$ outputs. We declare the $i$-th input of $A$ as the $i$-th input of $A\times B$ and the $i$-th input of $B$ as the $(Q(A) + i)$-th input of $A\times B$ (figure \ref{fig:figure1}).

\begin{figure}
\includegraphics[]{figure1.png}
\label{fig:figure1}
\caption{}
\end{figure}

If $Z(A) = Q(B)$ we get from $A$ and $B$ a circuit $B\circ A$ by switching the $i$-th output of $A$ to the $i$-th input of $B$.

A circuit of elements of $\mathcal{U}$ is a device, which is described inductively by the preceeding explanations.

If $\phi(A)$ ($\phi(B)$) is the function of circuit $A$ ($B$), then $\phi(A)\times \phi(B)$ is the function of $A\times B$, and $\phi(B)\circ \phi(A)$ for $Q(B) = Z(A)$  is the function of $B\circ A$.

The costs for the building blocks in $\mathcal{U}$ shall be defiend by the function $L : \mathcal{U} \rightarrow N \cup {0}$. We define:

\[
\begin{array}{lcr}
L(A\times B) & = & L(A) + L(B), \\
L(A\circ B) & = & L(A) + L(B).
\end{array}
\]

Hereby a price is assigned to each circuit.

Now, the task is the following: Given $f : S^n \rightarrow S^m$, find a circuit $A$ with $\phi(A) = f$ and
\[
L(A) = \min_{B \in \phi^{-1}(f)} \{ L(B). \}
\]

If $f$ does not fully map to $S^n$, but only to $R \subset S^n$, then the optimium shall be searched on $\cup_{g|R=f} \phi^{-1}(g)$. 

The task is generalized in an obvioius way, if $Q(f) = Z(F) = S^n$ and $f$, as it is often the case with finite automata, is determined just by a transformation of $S^n$, 

In order to solve this problem it appears advantageous to know relations, which allow to generate from an element $A \in \phi^{-1}(f)$ all the elements from the class $\phi^{-1}(f)$.

In the first two sections of this work a theory of interconnection of automata will be developed, as already sketched out in the description of the task.
\end{document}
